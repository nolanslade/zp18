\documentclass[22pt]{beamer}
\usepackage[orientation=portrait, size=custom, width=91.44, height=91.44,scale=1.2]{beamerposter} % 36in*2.5 = 90cm
\usepackage[absolute,overlay]{textpos}
\usepackage{bookmark} %pdflatex says to use this to avoid errors...
\usepackage{graphicx} %for including images
\graphicspath{{figs/}} %location of images
\usepackage{wrapfig} %wrap text around the images
\usepackage{listingsutf8}    %package for code environment; use this instead of verbatim to get automatic line break; use this instead of listings to get (•)
\usepackage{amsmath}
\usepackage{gensymb}
\usepackage[export]{adjustbox}
\usepackage[skins,theorems]{tcolorbox}
\usepackage{tikz}
\newcommand*\circled[1]{\tikz[baseline=(char.base)]{
            \node[shape=circle,draw,inner sep=2pt] (char) {#1};}}
\usepackage{array}
\usepackage{booktabs,adjustbox}
\usepackage{subcaption} 

%\mode<presentation>
%this doesn't seem to make any difference; leave for now for trying out
\usetheme{Berlin}
\definecolor{MacBlue}{rgb}{0.10196,0.22353,0.53725}
\definecolor{MacMaroon} {rgb}{0.47843, 0, 0.23137}
\definecolor{MacMaroon2} {rgb}{0.47451, 0, 0}
\definecolor{MacGray}{rgb}{0.50196,0.49804,0.51765}
\definecolor{MacMaroon3}{rgb}{00.47,0.2,0.31}
\definecolor{MacGold}{rgb}{1, 0.75,0.35}
\usecolortheme[named=MacMaroon2]{structure}
\setbeamertemplate{caption}[numbered]
\setbeamertemplate{navigation symbols}{}

\title{Virtual Reality in Experimental Economics}
\subtitle{Capstone Project: Computer Science} 
  \author[Shah, Brendon, Slade, Kishore \& Carette]{Aaska Shah, Kerala Brendon, Nolan Slade, Vyome Kishore, supervised by Dr.~Jacques Carette$^\dagger$ and Dr.~Stephanie Thomas \vspace{0.3cm} \newline \small \{shaha8, brendokh, sladenj, kishov, carette\}@mcmaster.ca, stephanie.thomas1@curtin.edu.au and Team}
  \institute[McMaster University]{$^\dagger$Department of Computing and Software, McMaster University

1280 Main St. W, Hamilton, Ontario, Canada L8S 4L8}
  \date{December 5th, 2018}

\begin{document}
%compile with pdflatex

%there is only one frame, because there is only one page; yeah, it's a poster
%textblock and block seem to work nicely to organize layout
\begin{frame}[fragile]

\begin{textblock}{2}(0.7,1)
\includegraphics[height=8.5cm]{englogo.png} % We can use CAS logo as well? 
\end{textblock}

\begin{textblock}{8}(4,1)
\titlepage
\end{textblock}

\begin{textblock}{7.25}(0.5,3.1)

%this needs help
\begin{block}{Project Overview}\newline
\begin{itemize}
\item \textbf{What}: develop virtual reality environments, as an extension of previous work done by the McMaster Decision Science Laboratory, to carry out experimental economics research.
\item \textbf{Why}: gain insight into how participants make decisions when they are impaired.
\item The focus of the simulations is a basic task.
\item Participants are paid cash for successful task completions.
\item Participants will be virtually impaired and required to choose between methods of treatment.
\item Simulation environments are designed in Unity; to be used with an HTC Vive. 
\end{itemize}
\end{block}

\begin{block}{Existing Simulation}\newline
\begin{itemize}
\item \textbf{Task}: move crates from a pile to the target block.
\item The environment does not scale to the Vive-equipped testing room (Figure \ref{fig:crate}).
\item No extensive customization of configuration variables.
\end{itemize}
\newline
\begin{figure}
  \includegraphics[height=17cm]{CrateOriginal.PNG}
  \caption{Experiment room}
\label{fig:crate}
\end{figure}
\end{block}


\begin{block}{Our Simulations}\newline
\begin{itemize}
\item The virtual environments and experiment structures will be highly customizable. 
\item Metrics will be tracked and persisted in a database for analysis.
\item Environments are designed according to the physical constraints of the Vive-equipped testing room (Figure \ref{fig:room}).
\end{itemize}
\newline
\begin{figure}
  \includegraphics[height=22cm]{NolanVRRoom.png}
  \caption{Experiment room}
\label{fig:room}
\end{figure}
\end{block}






%\begin{block}{Conclusions \& Future Work}
%\begin{itemize}
%\item Using the existing experiment and consulting with the McMaster Decision Science Laboratory, we have developed a plan to implement the two simulations as described.
%\item These simulations will allow the laboratory to run unique experiments using either of the simulations with specific configurations while collecting data in a SQL database.
%\item The simulations will be tailored to the Vive equipped test room so the overall experience is as realistic as possible.
%\end{itemize}
%\end{block}

\end{textblock}




\begin{textblock}{7.25}(8.25,3.1)

\begin{block}{Summary of Configuration Variables}\newline
\begin{itemize}
\item Cash obtained by the participant for completed task iterations.
\item Impairment types and their intensities.
\item Treatment methods, costs, and effectiveness.
\item Per-day settings such as duration and active impairment(s).
\end{itemize}
\end{block}

\begin{block}{Simulation One}\newline
\begin{itemize}
\item \textbf{Task}: repeatedly transport a
volume of liquid between a source and destination using a single hand-carried vessel.
\item \textbf{Goal}: maximize the total volume of liquid that successfully reaches the destination.
\end{itemize}

\begin{figure}
  \centering
  \includegraphics[height=17cm]{SimulationOne.png}
  \caption{First simulation environment}
\label{fig:simOne}
\end{figure}

%%Have to add these
 %Figure #. Taken from the night demo shows the application of transparency in animation
\end{block}



\begin{block}{Simulation Two}\newline
\begin{itemize}
\item \textbf{Task}: sort a set of three-dimensional shapes into separate containers by passing them through a filter that only permits one particular shape.
\item \textbf{Goal}: maximize the total number of shapes sorted into the correct container.
\end{itemize}
\begin{figure}
  \begin{subfigure}[b]{0.37\textwidth}
    \includegraphics[height=14cm]{shape_sorting_environment.png}
    \caption{Layout of simulation environment.}
  \end{subfigure}
 %
  \begin{subfigure}[b]{0.60\textwidth}
    \includegraphics[height=14cm]{shape_shorting_filter.png}
    \caption{Shape sorting filter.}
    \end{subfigure}
 \end{figure}
\end{block}


\begin{block}{Sound Interesting?}\newline
\begin{itemize}
\item Email us what you think of our project, and/or to participate in experimental trials.
\item Please CC our coordinator, Dr. Christopher Anand: anandc@mcmaster.ca
\item Experimental trials will likely be scheduled for February 2019.
\end{itemize}
\end{block}


\begin{block}{Acknowledgements}\newline
\begin{itemize}
\item David Cameron: \textit{Manager, McMaster Decision Science Laboratory}
\item Neil Buckley: \textit{Associate Professor, York University}
\item Courtney Sheppard: \textit{IT Advisor, McMaster Decision Science Laboratory}
\end{itemize}
\end{block}

% \begin{block}{References}
% \setbeamertemplate{bibliography item}{\insertbiblabel}
% \bibliographystyle{ieeetr}
% {\scriptsize
% \bibliography{bib}}
% \end{block}

\begin{comment}
%these aren't in any particular style, it's just the basic idea
\begin{block}{References}
\setbeamertemplate{bibliography item}{\insertbiblabel}
\bibliographystyle{ieeetr}
{\scriptsize
\bibliography{bib}}
\end{block}
\vspace{-1.8mm}
%will need some more graphics to thank the various people
\end{comment}
\end{textblock}


\end{frame}
\end{document}
